\documentclass[11 pt]{article}

%%%%% AMS Math packages and more %%%%%

\usepackage{amsmath}
\usepackage{amsfonts}
\usepackage{amssymb}
\usepackage{amsopn}
\usepackage{amsthm}
\usepackage{mathrsfs}
\usepackage{mathtools}
\usepackage{needspace}
\usepackage{extpfeil}
\usepackage{fullpage}
\usepackage[framemethod=tikz]{mdframed}
\usepackage[margin=0.3in]{geometry}

%%%%% Commutative Diagram Package %%%%%

\usepackage{tikz}
\usepackage{tikz-cd}
\usetikzlibrary{arrows}

%%%%% Analysis %%%%% 

\newcommand{\argmin}{\operatorname{\mathrm{argmin}}}
\newcommand{\argmax}{\operatorname{\mathrm{argmax}}}

%%%%% Symbols and expressions %%%%%

\newcommand{\1}{\operatorname{\mathbb I}}
\newcommand{\R}{\operatorname{\mathbb R}}
\newcommand{\N}{\operatorname{\mathbb N}}
\newcommand{\Z}{\operatorname{\mathbb Z}}
\newcommand{\F}{\operatorname{\mathbb F}}
\newcommand{\Q}{\operatorname{\mathbb Q}}
\newcommand{\C}{\operatorname{\mathbb C}}
\newcommand{\A}{\operatorname{\mathbb A}}
\newcommand{\PP}{\operatorname{\mathbb P}}
\newcommand{\ZNZ}[1]{\Z \! / {#1} \! \Z}
\newcommand{\QZ}{\Q \! / \! \Z}
\newcommand{\p}{\mathfrak p}
\renewcommand{\a}{\mathfrak a}
\renewcommand{\b}{\mathfrak b}
\renewcommand{\c}{\mathfrak c}
\newcommand{\dd}{\mathfrak d}
\newcommand{\m}{\mathfrak m}
\newcommand{\n}{\mathfrak n}
\renewcommand{\P}{\mathfrak P}
\newcommand{\q}{\mathfrak q}
\newcommand{\re}{\mathfrak{R}\mathrm{e}}
\newcommand{\im}{\mathfrak{I}\mathrm{m}}
\newcommand{\ima}{\mathrm{im} \, }
\newcommand{\cok}{\mathrm{coker} \, }
\newcommand{\idof}{\trianglelefteq}
\newcommand{\NNT}[1]{\mathrm{NNT}(#1)}
\newcommand{\NT}[1]{\mathrm{NT}(#1)}
\newcommand{\nil}[1]{\mathrm{Nil}(#1)}
\newcommand{\red}{\mathrm{red}}
\newcommand{\cha}[1]{\mathrm{ch}(#1)}
\newcommand{\Hom}[3]{\mathrm{Hom}_{#1}(#2,#3)}
\newcommand{\Der}[3]{\mathrm{Der}_{#1}(#2,#3)}
\newcommand{\der}[2]{\mathrm{Der}_{#1}(#2)}
\newcommand{\HOM}{\HH \!\!\mathit{om}}
\newcommand{\End}[2]{\mathrm{End}_{#1}(#2)}
\newcommand{\Mat}[1]{\mathbf{Mat}(#1)}
\newcommand{\Matmn}[3]{\mathbf{Mat}_{#1 \times #2}(#3)}
\newcommand{\spec}[1]{\mathrm{Spec} \!\left(#1 \right)}
\newcommand{\Spec}[1]{\mathbf{Spec} \!\left(#1 \right)}
\newcommand{\jac}[1]{\mathrm{Jac}(#1)}
\newcommand{\ass}[2]{\mathrm{Ass}_{#1}(#2)}
\newcommand{\ann}[2]{\mathrm{Ann}_{#1}(#2)}
\newcommand{\supp}[2]{\mathrm{supp}_{#1}(#2)}
\newcommand{\Supp}[1]{\mathrm{supp}(#1)}
\newcommand{\sing}[1]{\mathrm{Sing}(#1)}
\newcommand{\ins}[3]{\mathrm{Ins}_{#1,#2}\left(#3\right)}
\newcommand{\gr}[2]{\mathrm{gr}_{#1}(#2)}
\newcommand{\grR}[1]{\mathrm{gr}(#1)}
\newcommand{\defn}{\overset{def}{=}}
\newcommand{\st}{\quad \text{s.t.} \quad}
\renewcommand{\implies}{\quad \Longrightarrow \quad}
\newcommand{\aut}{\mathrm{Aut} \,}
\newcommand{\Aut}[2]{\mathrm{Aut}_{#1}(#2)}
\newcommand{\cyc}[1]{\Z \! / {#1} \! \Z }
\newcommand{\lenr}[2]{\ell_{#1} \! \left( #2 \right)}
\newcommand{\len}[1]{\ell \left( #1 \right)}
\newcommand{\coh}[1]{\mathfrak{Coh} \left( #1 \right)}
\newcommand{\qco}[1]{\mathfrak{Qco} \left( #1 \right)}
\newcommand{\LT}[1]{\mathrm{LT}(#1)}
\newcommand{\ar}[1]{\xrightarrow{#1}}
\newcommand{\tr}[1]{\mathrm{tr} \! \left( #1 \right)}
\newcommand{\Mod}[1]{{#1}\mathrm{-Mod}}
\newcommand{\MOD}{\mathrm{Mod}}
\newcommand{\id}[1]{\mathrm{id}_{#1}}
\newcommand{\rees}[1]{\mathcal R( #1 )}
\newcommand{\hgt}[1]{\mathrm{ht}(#1)}
\newcommand{\GL}[1]{\mathrm{GL}(#1)}
\newcommand{\PGL}[1]{\mathrm{PGL}(#1)}
\newcommand{\GLn}[2]{\mathrm{GL}_{#1}(#2)}
\newcommand{\ord}[1]{\mathrm{ord}(#1)}
\newcommand{\Ord}{\mathrm{ord}}
\newcommand{\zeros}[1]{\mathcal Z \! \left( #1 \right)}
\newcommand{\zerosp}[1]{\mathcal Z_+ \! \left( #1 \right)}
\newcommand{\ideal}[1]{\mathcal I \! \left( #1 \right)}
\newcommand{\Res}[2]{\mathrm{Res} \left( #1,#2 \right)}
\newcommand{\res}[2]{\mathrm{res}_{#1,#2}}
\newcommand{\ev}[1]{\mathrm{ev}_{#1}}
\newcommand{\PN}[1]{\mathbb P^{#1}}
\newcommand{\pspace}[2]{\mathbb P^{#1}(#2)}
\newcommand{\codim}[2]{\mathrm{codim}(#1,#2)}
\newcommand{\blowup}[2]{\mathrm{Bl}_{#1}(#2)}
\newcommand{\maxspec}[1]{\mathrm{MaxSpec} \left( #1 \right)}
\newcommand{\proj}[1]{\mathrm{Proj}\! \left( #1 \right)}
\newcommand{\Proj}[1]{\mathbf{Proj}\! \left( #1 \right)}
\newcommand{\distop}[1]{\mathcal D \!\left(#1\right)}
\newcommand{\distopl}[1]{\mathcal D_+ \!\left(#1\right)}
\newcommand{\AAA}{\mathscr A}
\renewcommand{\AA}{\mathcal A}
\newcommand{\BB}{\mathcal B}
\newcommand{\CC}{\mathcal C}
\newcommand{\DD}{\mathcal D}
\newcommand{\OO}{\mathscr O}
\newcommand{\EE}{\mathscr E}
\newcommand{\FF}{\mathscr F}
\newcommand{\GG}{\mathscr G}
\newcommand{\HH}{\mathscr H}
\newcommand{\II}{\mathscr I}
\newcommand{\JJ}{\mathscr J}
\newcommand{\KK}{\mathscr K}
\newcommand{\LL}{\mathscr L}
\newcommand{\MM}{\mathscr M}
\newcommand{\largewedge}{\mbox{\large $\wedge$}}
\newcommand{\colim}{\mathrm{colim}}
\newcommand{\pic}[1]{\mathrm{Pic}\! \left( #1  \right)}
\newcommand{\class}[1]{\mathrm{Cl} \! \left(#1 \right)}
\newcommand{\Div}[1]{\mathrm{div}\! \left( #1 \right)}
\newcommand{\prDiv}[1]{\mathrm{div}\! \left[ #1 \right]}
\newcommand{\WDIV}[1]{\mathrm{Div}_W\! \left( #1 \right)}
\newcommand{\PDIV}[1]{\mathrm{PDiv}_W\! \left( #1 \right)}
\newcommand{\WCl}[1]{\mathrm{Cl}_W\! \left( #1 \right)}
\newcommand{\CaDIV}[1]{\mathrm{Div}_{\mathrm{Ca}}\! \left( #1 \right)}
\newcommand{\PCaDIV}[1]{\mathrm{PDiv}_{\mathrm{Ca}}\! \left( #1 \right)}
\newcommand{\CaCl}[1]{\mathrm{CaCl} \left( #1 \right)}
\newcommand{\trivext}[2]{#1 \, \widehat{\oplus} \, #2}
\newcommand{\rank}[2]{\mathrm{rank}_{#1}(#2)}
\newcommand{\bdeg}[1]{\mathrm{bdeg} \, {#1}}
\newcommand{\Com}[1]{\mathrm{Com}({#1})}
\newcommand{\ComH}[1]{\mathrm{Com}_H({#1})}
\newcommand{\ComC}[1]{\mathrm{Com}^C({#1})}
\newcommand{\Ext}[4]{\mathrm{Ext}_{#1}^{#2}(#3,#4)}
\newcommand{\EXt}[4]{\overline{\mathrm{Ext}}_{#1}^{#2}(#3,#4)}
\newcommand{\Tor}[4]{\mathrm{Tor}_{#2}^{#1}(#3,#4)}
\newcommand{\TOr}[4]{\overline{\mathrm{Tor}}_{#2}^{#1}(#3,#4)}
%% Commands in Category Theory %%
\newcommand{\zero}{\mathbf 0}
\newcommand{\one}{\mathbf 1}
\newcommand{\two}{\mathbf 2}
\newcommand{\three}{\mathbf 3}
\newcommand{\op}{\mathrm{op}}
\newcommand{\Aa}{\operatorname{\mathbf A}}
\newcommand{\Cc}{\operatorname{\mathbf C}}
\newcommand{\Dd}{\operatorname{\mathbf D}}
\newcommand{\Ee}{\operatorname{\mathbf E}}
\newcommand{\Oo}{\operatorname{\mathbf O}}
\newcommand{\Pp}{\operatorname{\mathbf P}}
\newcommand{\Qq}{\operatorname{\mathbf Q}}
\newcommand{\dom}[1]{\mathrm{dom}(#1)}
\newcommand{\cod}[1]{\mathrm{cod}(#1)}
\newcommand{\Nat}[2]{\mathrm{Nat}(#1,#2)}
\newcommand{\Dom}{\mathrm{dom}}
\newcommand{\Cod}{\mathrm{cod}}
\newcommand{\ov}[1]{\overset{#1}{\longrightarrow}}
\newcommand{\dov}[2]{\overset{#1}{\underset{#2}{\rightrightarrows}}}
\newcommand{\Cat}{\mathbf{Cat}}
\newcommand{\Pre}{\mathbf{Pre}}
\newcommand{\AbCat}{\mathbf{Ab}\text-\mathbf{Cat}}
\newcommand{\Set}[1]{\mathbf{Set}_{#1}}
\newcommand{\SET}{\mathbf{Set}}
\newcommand{\SETp}{\mathbf{Set}_*}
\newcommand{\Finord}{\mathbf{Finord}}
\newcommand{\Fin}{\mathbf{Fin}}
\newcommand{\Mon}{\mathbf{Mon}}
\newcommand{\Grp}{\mathbf{Grp}}
\newcommand{\Rng}{\mathbf{Rng}}
\newcommand{\Ring}{\mathbf{Ring}}
\newcommand{\CRng}{\mathbf{CRng}}
\newcommand{\CRing}{\mathbf{CRing}}
\newcommand{\Ab}{\mathbf{Ab}}
\newcommand{\RMod}[1]{#1\text-\mathbf{Mod}}
\newcommand{\ModR}[1]{\mathbf{Mod}\text-#1}
\newcommand{\RModS}[2]{#1\text-\mathbf{Mod}\text-#2}
\newcommand{\Vect}[1]{\mathbf{Vect}_{#1}}
\newcommand{\Top}{\mathbf{Top}}
\newcommand{\Toph}{\mathbf{Toph}}
\newcommand{\Topp}{\mathbf{Top}_*}
\newcommand{\Topph}{\mathbf{Toph}_*}
\newcommand{\Sch}[1]{\mathbf{Sch}_{#1}}
\newcommand{\SCH}{\mathbf{Sch}}
\newcommand{\Sym}[2]{\mathrm{Sym}^{#1}(#2)}
\newcommand{\del}{\partial}

%%%%% Disposition commands %%%%%

\setlength{\parindent}{0 pt}
\setlength{\parskip}{15 pt}
\newcommand{\prg}
{

\vspace{12 pt} 

}
\renewcommand{\d}[1]{\ensuremath{ \displaystyle #1 }}

%%%%% Arrows %%%%%

\newcommand{\surj}{\xtwoheadrightarrow{\,}}
\newcommand{\inj}{\xhookrightarrow{\quad}}

%%%%% Titles %%%%%

\title{\textbf{Category Theory}}
\date{by Patrick Da Silva}
\author{Personal Notes}
\newcommand{\dateoftheday}[2]{ %[day, date]
\newpage
\begin{center}
{\Large {#1}, {#2}}
\end{center}
}

%%%%% New environments %%%%%

\newcounter{env} % Environment counter
\newcounter{chap} % Chapter counter
\newcounter{sect}
\newcounter{ax}

%%%%%%%%%%
%
% Adding nothing : nothing
% Adding t : environment + title
% Adding h : no labelling of the environment
% Adding th : environment + title + no labelling
%
%%%%%%%%%%


\newenvironment{abb} % AntiBreakBox
{
	\begin{mdframed}[bottomline=false,topline=false,leftline=false,rightline=false,nobreak,leftmargin=-10 pt,rightmargin=0 pt,skipabove=0 pt,skipbelow=0 pt]
}
{
	\end{mdframed}
}


\newenvironment{defin}	
{\par
	\addtocounter{env}{1}
	\noindent \begin{abb} \textbf{\underline{Definition \thechap.\theenv.}} \par
	\nopagebreak%
	\par
	\nopagebreak \itshape%
}
{%
	\par
	\end{abb}
}

\newenvironment{axiom}	
{\par
	\addtocounter{ax}{1}
	\noindent \begin{abb} \textbf{\underline{Axiom \theax.}} \par
	\nopagebreak%
	\par
	\nopagebreak \itshape%
}
{%
	\par
	\end{abb}
}


\newenvironment{definh}	
{\par
	\noindent \begin{abb} \textbf{\underline{Definition.}} \par
	\nopagebreak%
	\par
	\nopagebreak \itshape%
}
{%
	\par
	\end{abb}
}

\newenvironment{defint}[1]	
{\par
	\addtocounter{env}{1}
	\noindent \begin{abb} \textbf{\underline{Definition \thechap.\theenv.}} ({#1}) \par
	\nopagebreak%
	\par
	\nopagebreak \itshape%
}
{%
	\par
	\end{abb}
}

\newenvironment{thm}	
{\par
	\addtocounter{env}{1}
	\noindent \begin{abb} \textbf{\underline{Theorem \thechap.\theenv.}} \par
	\nopagebreak%
	\par
	\nopagebreak \itshape%
}
{%
	\par
	\end{abb}
}

\newenvironment{thmt}[1]	
{\par
	\addtocounter{env}{1}
	\noindent \begin{abb} \textbf{\underline{Theorem \thechap.\theenv.}} ({#1}) \par
	\nopagebreak%
	\par
	\nopagebreak \itshape%
}
{%
	\par
	\end{abb}
}

\newenvironment{thmh}	
{\par
	\noindent \begin{abb} \textbf{\underline{Theorem.}} \par
	\nopagebreak%
	\par
	\nopagebreak \itshape%
}
{%
	\par
	\end{abb}
}

\newenvironment{thmth}[1]	
{\par
	\noindent \begin{abb} \textbf{\underline{#1.}} \par
	\nopagebreak%
	\par
	\nopagebreak \itshape%
}
{%
	\par
	\end{abb}
}



\newenvironment{ubung}	
{\par
	\noindent \begin{abb} \textbf{\underline{Exercise.}} \par
	\nopagebreak%
	\par
	\nopagebreak%
}
{%
	\par
	\end{abb}
}

\newenvironment{remark}	
{\par
	\addtocounter{env}{1}
	\noindent \begin{abb} \textbf{\underline{Remark \thechap.\theenv.}} \par
	\nopagebreak%
	\par
	\nopagebreak%
}
{%
	\par
	\end{abb}
}

\newenvironment{remarkh}	
{\par
	\noindent \begin{abb} \textbf{\underline{Remark.}} \par
	\nopagebreak%
	\par
	\nopagebreak%
}
{%
	\par
	\end{abb}
}


\newenvironment{remarkt}[1]	
{\par
	\addtocounter{env}{1}
	\noindent \begin{abb} \textbf{\underline{Remark \thechap.\theenv.} (#1)} \par
	\nopagebreak%
	\par
	\nopagebreak%
}
{%
	\par
	\end{abb}
}


\newenvironment{ruleh}	
{\par
	\noindent \begin{abb} \textbf{\underline{Rule.}} \par
	\nopagebreak%
	\par
	\nopagebreak%
}
{%
	\par
	\end{abb}
}

\newenvironment{notation}	
{\par
	\addtocounter{env}{1}
	\noindent \begin{abb} \textbf{\underline{Notation \thechap.\theenv.}} \par
	\nopagebreak%
	\par
	\nopagebreak%
}
{%
	\par
	\end{abb}
}

\newenvironment{notationh}	
{\par
	\noindent \begin{abb} \textbf{\underline{Notation.}} \par
	\nopagebreak%
	\par
	\nopagebreak%
}
{%
	\par
	\end{abb}
}

\newenvironment{example}	
{\par
	\addtocounter{env}{1}
	\noindent \textbf{\underline{Example \thechap.\theenv.}} \par
	\nopagebreak%
	\par
	\nopagebreak \itshape%
}
{%
	\par
}

\newenvironment{exampleh}	
{\par
	\noindent \textbf{\underline{Example.}} \par
	\nopagebreak%
	\par
	\nopagebreak \itshape%
}
{%
	\par
}

\newenvironment{examplet}[1]	
{\par
	\addtocounter{env}{1}
	\noindent \textbf{\underline{Example \thechap.\theenv.}} ({#1}) \par
	\nopagebreak%
	\par
	\nopagebreak \itshape%
}
{%
	\par
}

\newenvironment{exampleth}[1]	
{\par
	\noindent \textbf{\underline{Example.}} ({#1}) \par
	\nopagebreak%
	\par
	\nopagebreak \itshape%
}
{%
	\par
}

\newenvironment{examples}	
{\par
	\addtocounter{env}{1}
	\noindent \textbf{\underline{Examples \thechap.\theenv.}} \par
	\nopagebreak%
	\par
	\nopagebreak \itshape%
}
{%
	\par
}

\newenvironment{examplesh}	
{\par
	\noindent \textbf{\underline{Examples.}} \par
	\nopagebreak%
	\par
	\nopagebreak \itshape%
}
{%
	\par
}

\newenvironment{examplest}[1]	
{\par
	\addtocounter{env}{1}
	\noindent \textbf{\underline{Examples \thechap.\theenv.}} ({#1}) \par
	\nopagebreak%
	\par
	\nopagebreak \itshape%
}
{%
	\par
}

\newenvironment{examplesth}[1]	
{\par
	\noindent \textbf{\underline{Examples.}} ({#1}) \par
	\nopagebreak%
	\par
	\nopagebreak \itshape%
}
{%
	\par
}





\newenvironment{prop}	
{\par
	\addtocounter{env}{1}
	\noindent \begin{abb} \textbf{\underline{Proposition \thechap.\theenv.}} \par
	\nopagebreak%It so turns out that if we write
	\par
	\itshape%
}
{%
	\par
	\end{abb}
}

\newenvironment{proph}	
{\par
	\noindent \begin{abb} \textbf{\underline{Proposition.}} \par
	\nopagebreak%It so turns out that if we write
	\par
	\itshape%
}
{%
	\par
	\end{abb}
}

\newenvironment{propt}[1]	
{\par
	\addtocounter{env}{1}
	\noindent \begin{abb} \textbf{\underline{Proposition \thechap.\thethm.}} ({#1}) \par
	\nopagebreak%
	\par
	\itshape%
}
{%
	\par
	\end{abb}
}

\newenvironment{lem}	
{\par
	\addtocounter{env}{1}
	\noindent \begin{abb} \textbf{\underline{Lemma \thechap.\theenv.}} \par
	\nopagebreak%
	\par
	\itshape%
}
{%
	\par
	\end{abb}
}


\newenvironment{lemh}	
{\par
	\noindent \begin{abb} \textbf{\underline{Lemma.}} \par
	\nopagebreak%
	\par
	\itshape%
}
{%
	\par
	\end{abb}
}


\newenvironment{lemt}[1]	
{\par
	\addtocounter{env}{1}
	\noindent \begin{abb} \textbf{\underline{Lemma \thechap.\theenv.}} ({#1}) \par
	\nopagebreak%
	\par
	\itshape%
}
{%
	\par
	\end{abb}
}

\newenvironment{cor}
{\par
	\addtocounter{env}{1}
	\noindent \begin{abb} \textbf{\underline{Corollary \thechap.\theenv.}} \par
	\nopagebreak%
	\par
	\itshape%
}
{%
	\par
	\end{abb}
}

\newenvironment{corh}
{\par
	\noindent \begin{abb} \textbf{\underline{Corollary.}} \par
	\nopagebreak%
	\par
	\itshape%
}
{%
	\par
	\end{abb}
}

\newenvironment{cort}[1]
{\par
	\addtocounter{env}{1}%
	\noindent \begin{abb} \textbf{\underline{Corollary \thechap.\theenv.}} ({#1}) \par
	\nopagebreak%
	\par
	\itshape%
}
{%
	\par
	\end{abb}
}

\renewenvironment{proof}
{\par
	\begin{mdframed}[skipabove=12 pt,bottomline=false,topline=false,rightline=false]%
	\noindent \textit{\textbf{Proof.}} 
}
{%
	\end{mdframed}\par
	\begin{flushright} \vspace{-35 pt} $\square$ \end{flushright}
	\vspace{-20 pt}
}


%%%%% Labels made out of counters %%%%%

\makeatletter

\newcommand*{\labelling}[1] % To create a label, simply type \labelling{labelnamehere}.
{%
  \edef\@currentlabel{\csname theenv\endcsname}% Store current counter value in \@currentlabel
  \label{#1}% Store label

  \edef\@currentlabel{\csname thechap\endcsname}% Store current counter value in \@currentlabel
  \label{#1chap}% Store label
}

\makeatother

\newcommand{\lref}[1]{\ref{#1chap}.\ref{#1}} % To reference to it, simply use \lref{labelnamehere} (lref : label reference).

\newcommand{\newchap}[1]{ 
\newpage
\phantom{x}
\newpage
\begin{center}
	\textbf{\Large \underline{#1}}
\end{center}
\addtocounter{chap}{1}
\setcounter{sect}{0}
\setcounter{env}{0}
}

\newcommand{\newsect}[1]{
	\addtocounter{sect}{1}
	{\large \thesect. \underline{#1}}
}




%%%%%%%%%%%%%%%%%%%%%%%%%%%%%%%%%%%%%%%%
%%%%% SETTING LABELS WITH COUNTERS %%%%%
%%%%%%%%%%%%%%%%%%%%%%%%%%%%%%%%%%%%%%%%
%% To set where the label should refer to
%% with a customized label name, 
%% if some theorem uses the counter ``thm'' 
%% as its numbering, in the theorem 
%% (or lemma/corollary/proposition) 
%% environment, write
%% %% %% %% %% %% %% %% %% %% %% %%
%%
%% \labelling{labelnamehere}{thm}
%%
%% %% %% %% %% %% %% %% %% %% %% %%
%% right after the \begin. To refer to 
%% this label, simply type \ref{labelnamehere}
%% where the reference should be.
%% Only use the same name once, otherwise 
%% there will be errors with \label.
%%%%%%%%%%%%%%%%%%%%%%%%%%%%%%%%%%%%%%%%
%% Environment counters : 
%% lem : thm
%% cor : thm
%% thm : thm
%% defin : def
%% example : ex
%% examples : none for now
%%%%%%%%%%%%%%%%%%%%%%%%%%%%%%%%%%%%%%%%





\begin{document}
\begin{definh}
    A (mathematical) \textbf{time series} is a function $f: \Z \to \R$. In practice, time series are represented over a finite time interval, taking values at equally separated discrete time steps. We can turn such a time series encountered in practice by mapping the given values to $\Z$ in a linear fashion, and extending the time series to the left by its first value, and to the right by its last value.  
\end{definh}

\begin{definh}
    Let $k \ge 1$ be an integer and $x$ denote a time series and denote its value at time $t \in \Z$ by $x_t$. The \textbf{moving average of window $k$} is defined as the following time series:
    \[
        M^k(x)_t \defn \sum_{i = -\infty}^{\infty} x_{t-i} m_i^k
    \]
    where
    \[
        m_i^k \defn
        \begin{cases}
            \frac 1{2k+1}   & \text{ if } |i| \le k \\
            0               & \text{ if } |i| > k.
        \end{cases}
    \]
    If we denote by $x * y$ the \textbf{convolution} operation of two time series:
    \[
        (x * y)_t = \sum_{i=-\infty}^{\infty} x_{t-i} y_i,
    \]
    then $M^k(x)_t = (x * m^k)_t$ where $m^k$ is the time series defined above.
\end{definh}

We want to study what happens when we compute the moving average of a moving average. 

\begin{proph}
    Convolution is commutative and associative, i.e. for three time series $x,y,z : \Z \to \R$, we have
    \[
            x * y = y * z, \qquad (x * y) * z = x * (y * z).
    \]
\end{proph}

\begin{proof}
    Let us first consider commutativity. For $t \in \Z$, we have 
    \begin{gather*}
        \begin{aligned}
            (x * y)_t
                & = \sum_{i=-\infty}^{\infty} x_{t-i} y_i \\
                & = \underset{i+j = t}{\sum_{(i,j) \in \Z^2}} x_j y_i \\
                & = \underset{i+j = t}{\sum_{(i,j) \in \Z^2}} x_i y_j \\
        \end{aligned}
    \end{gather*}
    The latter is symmetric in $x$ and $y$, so the operation is commutative. As for associativity,
    \begin{gather*}
        \begin{aligned}
            ((x * y) * z)_t 
                & = \underset{\ell+k = t}{\sum_{(\ell,k) \in \Z^2}} (x * y)_{\ell} z_k \\
                & = \underset{\ell+k = t}{\sum_{(\ell,k) \in \Z^2}} \left( \underset{i+j = \ell}{\sum_{(i,j) \in \Z^2}} x_i y_j  \right) z_k \\
                & = \underset{i+j+k = t}{\sum_{(i,j,k) \in \Z^3}} x_i y_j z_k.
        \end{aligned}
    \end{gather*}
    The latter expression is symmetric in $x$,$y$ and $z$, so since the operation is also commutative, we have
    \[
        (x * y) * z = (y * z) * x = x * (y * z). 
    \]        
\end{proof}

\begin{corh}
    Let $k,\ell \ge 1$ be two positive integers and $x$ a time series. The moving average of window $k$ of the moving average of window $\ell$ is a convolution with a fixed time series:
    \[
        M^k(M^{\ell}(x))_t = M^{\ell}(M^k(x))_t = x * m^{\ell,k}
    \]
    with 
    \[
        m^{\ell,k}_t \defn \frac 1{(2k+1)(2\ell+1)} \left(
            \begin{cases}
                2 \min \{\ell, k\} + 1                      & \text{ if } |t| \le |\ell - k|                \\
                2 \min \{\ell, k\} + 1 - (|t| - |\ell - k|) & \text{ if } |\ell - k| \le |t| \le \ell + k   \\
                0                                           & \text{ if } |t| > |\ell - k|. 
            \end{cases}
        \right)
    \]
\end{corh}

\begin{proof}
    By the properties of convolution, we have
    \[
        M^k(M^{\ell}(x)) = (x * m^k) * m^{\ell} = x * (m^k * m^{\ell}).
    \]
    By the commutativity of convolution, we already deduce that $M^k(M^{\ell}(x)) = M^{\ell}(M^k(x))$ and that $m^{\ell,k} \defn m^k * m^{\ell}$ is the time series we are looking for. \\

    Given $t \in \Z$, we want to compute the sum 
    \[
        m^{\ell,k}_t = \underset{i+j = t}{\sum_{(i,j) \in \Z^2}} m_i^k m_j^{\ell}.
    \]
    We have $i+j = t$ if and only if $(-i) + (-j) = -t$ and $m_i^k m_j^k = m_{-i}^k m_{-j}^k$, from which we deduce that $m^{\ell,k}_t = m^{\ell,k}_{-t}$. So it suffices to compute $m^{\ell,k}_t$ for $t \ge 0$. 

    Without loss of generality, assume $\ell \ge k$. For $0 \le t \le \ell - k$, since $|i| \le k$ implies $|t-i| \le |t| + |i| = (\ell -k) + k = \ell$, we have
    \[
        m^{\ell,k}_t = \sum_{i=-\infty}^{\infty} m_i^k m_{t-i}^{\ell} = \sum_{i=-k}^k m_{t-i}^{\ell} =\sum_{i=-k}^k \frac 1{2k+1} = 1.
    \]
    Now suppose $\ell - k \le t \le \ell + k$. Our objective is to show that
    \begin{gather*}
        \begin{aligned}
            (2k+1)(2\ell+1) m^{\ell,k}_t 
                & = 2 \min\{\ell,k\} + 1 - (|t| - |\ell - k|)   \\
                & = 2k + 1 - (t - (\ell - k))                   \\
                & = k + 1 + (\ell - t)                          \\
        \end{aligned}
    \end{gather*}
    Suppose $t \le \ell$. Since $\ell \ge k$, we know that $|t - k| \le \ell$ (because $0 \le k,t \le \ell$), thus $-\ell \le t-k \le t \le \ell$. Also, $\ell - k \le t$ implies $\ell \le t+k$. Therefore,
    \begin{gather*}
        \begin{aligned}
        (2k+1)(2\ell+1) m^{\ell,k}_t 
            & = \sum_{i=-\infty}^{\infty} ((2k+1) m_i^k) ((2\ell+1)m_{t-i}^{\ell})                \\
            & = \sum_{i=-k}^k (2\ell+1) m_{t-i}^{\ell}                                            \\
            & = \sum_{i=-k}^k (2\ell+1)m_{t+i}^{\ell}                                             \\
            & = \sum_{i=t-k}^{t+k} (2\ell+1)m_i^{\ell}                                            \\
            & = \sum_{i=t-k}^t (2\ell+1)m_i^{\ell} + \sum_{i=t+1}^{t+k} (2\ell+1)m_i^{\ell}       \\
            & = (k+1) + (\ell - t).
        \end{aligned}
    \end{gather*}
    Now suppose $t \ge \ell$. Since $t \le \ell + k$, we have $t - k \le \ell$. This implies
    \begin{gather*}
        \begin{aligned}
            (2k+1)(2\ell+1) m^{\ell,k}_t 
                & = \sum_{i=t-k}^{t+k} (2\ell+1) m_i^{\ell}   \\
                & = \sum_{i=t-k}^t (2\ell+1)m_i^{\ell} + \underbrace{\sum_{i=t+1}^{t+k} (2\ell+1) m_i^{\ell}}_{= 0} \\
                & = \sum_{i=t-k}^{\ell} (2\ell+1) m_i^{\ell}    \\
                & = \ell - (t-k) + 1 = (k+1) + (\ell - t).
        \end{aligned}
    \end{gather*}
    Finally, if $t > \ell + k$, then $t-k > \ell$, which means that
    \[
        (2k+1)(2\ell+1)m^{\ell,k}_t = \sum_{i=t-k}^{t+k} (2\ell+1) m_i^{\ell} = 0.
    \]
\end{proof}

\begin{proph}
    Let $x,y : \Z \to \R$ be two time series, and suppose that at least one of $x$ or $y$ takes only finitely many non-zero values. Then
    \[
        \sum_{t=-\infty}^{\infty} (x * y)_t = \left( \sum_{t=-\infty}^{\infty} x_t \right) \left( \sum_{t=-\infty}^{\infty} y_t \right).
    \]
\end{proph}

\begin{proof}
    We can re-write the expression on the left:
    \begin{gather*}
        \begin{aligned}
            \sum_{t=-\infty}^{\infty} m_t^{\ell,k} 
                & = \sum_{t=-\infty}^{\infty} \sum_{i=-\infty}^{\infty} x_i y_{t-i} \\
                & = \sum_{i=-\infty}^{\infty} \sum_{j=-\infty}^{\infty} x_i y_j \\
                & = \left( \sum_{i=-\infty}^{\infty} x_i \right) \left( \sum_{j=-\infty}^{\infty} y_j \right) \\
        \end{aligned}
    \end{gather*}
    The re-indexing of the sum is just a change of variables ($j=t-i$) and the last equality is justified by the fact that one of the sum is finite.
\end{proof}

\begin{corh}
    Let $k,\ell \ge 1$ be two positive integers and $x$ be a time series. Then
    \[
        \sum_{t=-\infty}^{\infty} M^k(x)_t = \sum_{t=-\infty}^{\infty} x_t.
    \]
    In particular, we have
    \[
        \sum_{t=-\infty}^{\infty} m_t^k = 1 = \sum_{t=-\infty}^{\infty} m_t^{\ell,k}.
    \]
\end{corh}

\begin{proof}
    We begin by proving that $\sum_{t=-\infty}^{\infty} m_t^k = 1$, which is a simple computation~:
    \[
        \sum_{t=-\infty}^{\infty} m_t^k = \sum_{t=-k}^k \frac 1{2k+1} = \frac 1{2k+1} \left(2k+1\right) = 1.
    \]
    As for the main result, noting that $M^k(x) = x * m^k$, we see that
    \[
        \sum_{t=-\infty}^{\infty} M^k(x)_t = \left(\sum_{t=-\infty}^{\infty} x_t\right)\left(\sum_{t=-\infty}^{\infty} m_t^k\right) = \sum_{t=-\infty}^{\infty} x_t.
    \]
    The last result follows from the fact that $m^{\ell,k}_t = M^{\ell}(m^k_t)$.
\end{proof} 

\end{document}
